\documentclass[a4paper,english]{article}
\usepackage{a4wide}
\usepackage{babel}
\usepackage{verbatim}

\usepackage{changepage}

\usepackage[bookmarksopen,bookmarksnumbered]{hyperref}

\usepackage[fancyhdr]{latex2man}



\newcommand{\thecmd}{gourd.toml}
\newcommand{\thecommand}{GOURD.TOML}
\newcommand{\mansection}{1}
\newcommand{\mansectionname}{File Formats Manual}
\newcommand{\mandate}{5 MAY 2024}
\setDate{5 MAY 2024}
\setVersionWord{Version:}
\setVersion{1.29}


\fancyhead[L, R]{\textit{\thecommand}(\mansection)}
\fancyhead[C]{\textsc{\mansectionname}}
\fancyfoot[C]{\mandate}
\fancyfoot[R]{\thepage}
\renewcommand{\headrulewidth}{0pt}

\renewcommand{\Prog}[1]{\textbf{#1}}                      % Program name

\renewenvironment{abstract}{\noindent\bfseries NAME\normalfont\vspace{0.2cm}}{}

\renewenvironment{Name}[5]{
\title{#5}
\author{#3}
\date{\@LM@Date\\{\small Version \@LM@Version}}
\begin{abstract}
}{
\end{abstract}
}

\renewenvironment{Description}[1][]{
\begin{list}{}{
 \ifthenelse{\equal{#1}{}}{
   % optional argument not given
     \labelwidth\z@ \itemindent-\leftmargin
     \let\makelabel\descriptionlabel
     \renewcommand{\makelabel}[1]{\hspace\labelsep\normalfont\bfseries##1}
  }{
   % optional argument given
   \settowidth{\labelwidth}{\normalfont\bfseries#1}
   \setlength{\leftmargin}{\labelwidth}
   \addtolength{\leftmargin}{\labelsep}
   \renewcommand{\makelabel}[1]{\normalfont\bfseries##1\hfil}
  }}
}{
\end{list}
}

\usepackage{titlesec}
\titleformat{\section}
  {\normalfont\normalsize\bfseries}{\thesection}{1em}{}

\titleformat{\subsection}
  {\normalfont\normalsize\bfseries}{\thesubsection}{1em}{}


\usepackage{mathspec}
\setmainfont[Mapping=tex-text, FakeBold=1]{Linux Libertine O}
\setmathfont(Digits,Greek,Latin)[Numbers=OldStyle, FakeBold=1]{Linux Libertine O}

\begin{document}
  \pagestyle{fancy}

  \begin{Name}{5}{gourd.toml}{Test}{File Formats Manual}{gourd.toml}
%@% IF LATEX %@%
\begin{adjustwidth}{18pt}{0pt}
%@% END-IF %@%

      \Prog{gourd.toml} \- An experimental setup file for the Gourd experiment scheduler.

%@% IF LATEX %@%
\end{adjustwidth}
%@% END-IF %@%
  \end{Name}

%@% IF LATEX %@%
\begin{adjustwidth}{18pt}{0pt}
%@% END-IF %@%

  \section{DESCRIPTION}

      \noindent \Prog{gourd.toml} is a declarative configuration file that,
      when provided to \Prog{gourd(1)}, can be used to create a new \emph{experiment}.
      The rules for typesetting the file follow the standard TOML format.

      By default, the filename is \File{gourd.toml}.

  \section{VALUE TYPES}

    \begin{Description}[Types]\setlength{\itemsep}{0cm}
        \item[string] A string of characters surrounded by "".
        \item[path] A file path surrounded by "".
        \item[list of T] A list of values of type T surrounded by [].
    \end{Description}

  \section{GLOBAL CONTEXT}

    These should be put at the beginning of the file before defining any sections.
    Options surrounded by brackets are optional and can be omitted, the default value is show in brackets in the description.

    \begin{Description}[Options]\setlength{\itemsep}{0cm}
        \item[\Opt{output\_path} = path]
            Where to store the stdout files from the \emph{programs}.
        \item[\Opt{metrics\_path} = path]
          Where to store the metrics for \Prog{gourd(1)} \Arg{status}.
        \item[\Opt{experiments\_folder} = path]
            Where to store state of previously ran experiments.
        \item[\Opt{wrapper?} = string]
            Defines the path to the \File{gourd-wrapper} binary.
            The default value is \Arg{gourd-wrapper}.
    \end{Description}


%    \section{SLURM}
%
%      An optional subsection of the config file, beginning with the \Arg{[slurm]} header.
%
%      \begin{Description}[Options]\setlength{\itemsep}{0cm}
%          \item[\Opt{experiment\_name} = string] The SLURM internal experiment name.
%          \item[TODO] TODO.
%      \end{Description}


    \section{PROGRAMS}

      Multiple programs can be specified.
      A program represents a compiled algorithm and is a combination of a binary file and parameters.
      Each program begins with \Arg{[programs.program-name]}, where \Arg{program-name} can be any unique name.

      \begin{Description}[Options]\setlength{\itemsep}{0cm}
          \item[\Opt{binary} = path]
            Path to the program executable.
          \item[\Opt{arguments?} = list of string]
            Arguments to be passed to the executable.
            The default is no arguments.
          \item[\Opt{afterscript?} = see afterscript]
            The configuration for running a short script after program completion.
      \end{Description}

    \section{INPUTS}

      A \Prog{gourd(1)} experiment consists of a cross-product mapping between programs
      and inputs.
      The experiment created from a \File{gourd.toml} file runs every combination
      of program and input in the file.

      Multiple inputs can be specified.
      Each input begins with \Arg{[inputs.input-name]} where \Arg{input-name} can be any unique name.
      \textbf{The string} \Arg{\_glob\_} \textbf{is reserved and cannot be used.}
      Each input contains the following keys:

      \begin{Description}[Options]\setlength{\itemsep}{0cm}
          \item[\Opt{input?} = path]
            Path to a file, the contents of which are passed to the program as standard input.
            By default, no standard input is provided.
          \item[\Opt{arguments?} = list of string]
            Additional command-line arguments to be passed to the program.
            The input arguments are appended to the programs arguments.
            By default, no additional arguments.
      \end{Description}


      \subsection{GLOBS}

        Globs can be applied to arguments and conveniently reference multiple files.

          If an argument starts with \Arg{glob|}, it will be treated as a glob.
          The input will be instantiated for every match of the provided glob.

            Example:

      \begin{verbatim}
[inputs.testrun1]
arguments = ["-f1", "=glob=./inputs/*.in", "-f2", "=glob=./input2/*.in"]
      \end{verbatim}

            Given that the current directory contains the files
            \File{input/1.in}, \File{input/2.in}, \File{input2/test.in}, the glob
            expands to the following experiment inputs:

      \begin{verbatim}
[inputs.testrun1_glob_0]
arguments = ["-f1", "./inputs/1.in", "-f2", "./input2/test.in"]

[inputs.testrun1_glob_1]
arguments = ["-f1", "./inputs/2.in", "-f2", "./input2/test.in"]
      \end{verbatim}


  \section{SEE ALSO}
      \Prog{gourd(1)}

  \section{AUTHORS}
    Lukáš Chládek <\Email{l@chla.cz}>\\[0.1cm]\MANbr
    Rūta Giedrytė <\Email{r.giedryte@student.tudelft.nl}>\\[0.1cm]\MANbr
    Ανδρέας Τσατσάνης <\Email{a.tsatsanis@student.tudelft.nl}>\\[0.1cm]\MANbr
    Mikołaj Gazeel <\Email{m.j.gazeel@student.tudelft.nl}>\\[0.1cm]\MANbr
    Jan Piotrowski <\Email{j.p.piotrowski@student.tudelft.nl}>
%@% IF LATEX %@%
\end{adjustwidth}
%@% END-IF %@%

\end{document}
