\documentclass[a4paper,english]{article}
\usepackage{a4wide}
\usepackage{babel}
\usepackage{verbatim}

\usepackage{changepage}

\usepackage[bookmarksopen,bookmarksnumbered]{hyperref}

\usepackage[fancyhdr]{latex2man}

\usepackage{xspace}



\newcommand{\thecmd}{gourd}
\newcommand{\thecommand}{GOURD}
\newcommand{\mansection}{1}
\newcommand{\mansectionname}{DelftBlue Tools Manual}
\newcommand{\mandate}{5 MAY 2024}
\setDate{5 MAY 2024}
\setVersionWord{Version:}
\setVersion{1.29}


\fancyhead[L, R]{\textit{\thecommand}(\mansection)}
\fancyhead[C]{\textsc{\mansectionname}}
\fancyfoot[C]{\mandate}
\fancyfoot[R]{\thepage}
\renewcommand{\headrulewidth}{0pt}

\renewcommand{\Prog}[1]{\textbf{#1}}                      % Program name

\renewenvironment{abstract}{\noindent\bfseries NAME\normalfont\vspace{0.2cm}}{}

\renewenvironment{Name}[5]{
\title{#5}
\author{#3}
\date{\@LM@Date\\{\small Version \@LM@Version}}
\begin{abstract}
}{
\end{abstract}
}

\renewenvironment{Description}[1][]{
\begin{list}{}{
 \ifthenelse{\equal{#1}{}}{
   % optional argument not given
     \labelwidth\z@ \itemindent-\leftmargin
     \let\makelabel\descriptionlabel
     \renewcommand{\makelabel}[1]{\hspace\labelsep\normalfont\bfseries##1}
  }{
   % optional argument given
   \settowidth{\labelwidth}{\normalfont\bfseries#1}
   \setlength{\leftmargin}{\labelwidth}
   \addtolength{\leftmargin}{\labelsep}
   \renewcommand{\makelabel}[1]{\normalfont\bfseries##1\hfil}
  }}
}{
\end{list}
}

\usepackage{titlesec}
\titleformat{\section}
  {\normalfont\normalsize\bfseries}{\thesection}{1em}{}

\titleformat{\subsection}
  {\normalfont\normalsize\bfseries}{\thesubsection}{1em}{}


\usepackage{mathspec}
\setmainfont[Mapping=tex-text, FakeBold=1]{Linux Libertine O}
\setmathfont(Digits,Greek,Latin)[Numbers=OldStyle, FakeBold=1]{Linux Libertine O}

\begin{document}
	\pagestyle{fancy}


%  \begin{adjustwidth}

  \begin{Name}{1}{gourd}{Test}{DelftBlue Tools Manual}{Gourd - Pumpkin}

      \Prog{gourd} - A tool for scheduling parallel runs for algorithm comparisons.

  \end{Name}
  \section{SYNOPSIS}

    \Prog{gourd} \oArg{command} \oOptArg{-c}{ filename} \oOpt{-d} \oOpt{-h} \oOpt{-s} \oOpt{-v|-vv}

  \section{DESCRIPTION}

    \Prog{gourd} is a tool that schedules parallel runs for algorithm comparisons.
    Given the parameters of the experiment, a number of test datasets, and algorithm implementations to compare,
    \Prog{gourd} runs the experiment in parallel and provides many options for processing its results.
    While originally envisioned for the DelftBlue supercomputer at Delft University of Technology,
    \Prog{gourd} can replicate the experiment on any cluster computer with the \Prog{Slurm} scheduler,
    on any UNIX-like system, and on Microsoft Windows.

    \section{OPTIONS}

    The following options apply to all \Prog{gourd} commands.

    \begin{Description}[Options]\setlength{\itemsep}{0cm}
    \item[\OptArg{-c}{ filename}, \OptArg{--config}{ filename}]
    Tell \Prog{gourd} to use the given filename as \File{gourd.toml}, the configuration
    file that defines the experimental setup.
    By default, the file is expected in the current working directory at \File{./gourd.toml}.
    \item[\Opt{-d}, \Opt{--dry-run}]
    Run \Prog{gourd} in dry-run mode, printing all operations (such as writing to files or scheduling runs)
    without executing them.
    \item[\Opt{-h}, \Opt{--help}]
    Display usage instructions for the \Prog{gourd} utility or any of its commands.
    \item[\Opt{-s}, \Opt{--script}]
    Tell \Prog{gourd} to use a script-friendly interface, that is, one that does not use
    interactive user prompts.
    \item[\Opt{-v}, \Opt{-vv}, \Opt{--verbose}]
    Run \Prog{gourd} with verbose output, where \Opt{-vv} enables even more logging.
    \end{Description}

    \section{COMMANDS}

    Using \Prog{gourd} is as simple as invoking one of its commands, such as `gourd status'.
    The following is a summary of available commands.

    \begin{Description}[Commands]\setlength{\itemsep}{0cm}
        \item[\Prog{gourd} \Arg{run}]
        Create an experiment from configuration and runs it on \Prog{Slurm} or the local machine.
        \item[\Prog{gourd} \Arg{init}]
        Set up a template of an experiment configuration.
        \item[\Prog{gourd} \Arg{status}]
        Display the status of an experiment that was run.
        \item[\Prog{gourd} \Arg{postprocess}]
        Run postprocessing jobs.
        \item[\Prog{gourd} \Arg{analyse}]
        Output metrics of completed runs.
        \item[\Prog{gourd} \Arg{version}]
        Show the software version.
    \end{Description}

    \subsection{GOURD RUN}

        \subsubsection{Summary}
        The \Prog{gourd} \Arg{run} command uses the provided configuration and
        runs an experiment.
        Using either \Arg{local} or \Arg{slurm}, it is possible for the execution
        to run on the local machine, or be scheduled using Slurm on a cluster computer.
        Using Slurm, additional configuration arguments are required; see \Prog{gourd.toml}(5).

        \subsubsection{Synopsis}
        \Prog{gourd} \Arg{run} \Arg{slurm}|\Arg{local} \oOptArg{-c}{ filename} \oOpt{-h} \oOpt{-s} \oOpt{-v|-vv}

    \subsection{GOURD INIT}

        \subsubsection{Summary}
        The \Prog{gourd} \Arg{init} command creates an experimental configuration.
        Configurations are represented as TOML files.
        A template configuration, \File{gourd.toml}, is placed in a Git repository.

        \subsubsection{Synopsis}
        \Prog{gourd} \Arg{init} \oOpt{-d} \oOptArg{-D}{ directory} \oOpt{-h} \oOpt{-s} \oOpt{-v|-vv}

        \subsubsection{Options}
        \begin{Description}[Options]\setlength{\itemsep}{0cm}
        \item[\OptArg{-D}{ directory}]
        Uses the given directory name as the repository root.
        \end{Description}

    \subsection{GOURD STATUS}

    \subsubsection{Summary}
    The \Prog{gourd} \Arg{status} command displays the status of an existing experiment,
    that is, one that has been created by \Prog{gourd} \Arg{run}, but not necessarily
    one that has fully executed.
    This command can also display detailed status of an individual job using the \Opt{-i} flag.

    \subsubsection{Synopsis}
    \Prog{gourd} \Arg{status} \oArg{experiment_id} \oOpt{-d} \oOpt{-h} \oOptArg{-i}{ run_id} \oOpt{-s} \oOpt{-v|-vv}

    \subsubsection{Options}
    \begin{Description}[Options]\setlength{\itemsep}{0cm}
    \item[\Arg{experiment_id}]
    The ID of an experiment to show the status of. By default, this is the most recent experiment.
    \item[\OptArg{-i}{ run_id}]
    The ID of a run to show detailed information of.
    \end{Description}


  \section{FILES}

    \begin{Description}[Files]\setlength{\itemsep}{0cm}
        \item[\File{gourd.toml}] A configuration file containing the experiment details. See \Prog{gourd.toml}(5).
        \item[\File{<experiment-dir>/<experiment-number>.lock}] A file containing the runtime data of the experiment.
    \end{Description}

  \section{SEE ALSO}

      \Prog{gourd.toml}(5)

  \section{AUTHORS}
    Lukáš Chládek <\Email{l@chla.cz}>\\[0.1cm]\MANbr
    Rūta Giedrytė <\Email{r.giedryte@student.tudelft.nl}>\\[0.1cm]\MANbr
    Andreas Tsatsanis <\Email{a.tsatsanis@student.tudelft.nl}>\\[0.1cm]\MANbr
    Mikołaj Gazeel <\Email{m.j.gazeel@student.tudelft.nl}>\\[0.1cm]\MANbr
    Jan Piotrowski <\Email{j.p.piotrowski@student.tudelft.nl}>

%  \section{KNOWN ISSUES}
%
%      The functionality of this software is not yet implemented.

%  \end{adjustwidth}

\end{document}
