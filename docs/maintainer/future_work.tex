\section{Future work}

\subsection{Removing the installer}

The entire installer template is contained in \texttt{src/resources/install.sh}.

It is quite easy to extend this installer with new files or paths. One option which
we considered a nice addition would be to skip installing parts of the application
on the request of the user (for example manpages).

Finally, as outlined in the beginning of the section, the biggest improvement
would be to disable the installer and distribute \gourd\ via system specific libraries.

It would be ideal if this process was included in \texttt{build.rs} via optional features.

We have identified the following packages as priorities to make \gourd\ accessible
to the largest user base:
\begin{itemize}
  \item A Debian \texttt{.deb} package. This enables Ubuntu packaging as well.
  \item A homebrew package script. This enables MacOS.
  \item A ArchLinux \texttt{PKGBUILD} file. This enabled distribution for ArchLinux.
  \item A RHEL \texttt{.rpm} package. This enables Fedora packaging as well.
  \item A \texttt{flake.nix} file for NixOS.
  \item A Windows \texttt{.msi} style installer.
\end{itemize}

\subsection{Memory hotspot analysis}

An originally planned feature was to add \texttt{kcachegrind} like analysis for
algorithms to identify memory hotspots.

We believe that this feature can be implemented by additions to the wrapper.

Currently the wrapper is a very simple program (on purpose) but an option may be
added to it to run, for example, \texttt{kcachegrind} on the ran program
and output, alongside all of the other metrics, the hotspot analysis.

\subsection{The wrapper errors}

In some usage cases it may happen that \texttt{gourd\_wrapper} fails
instead of \texttt{gourd}.

For example, if a file that \texttt{gourd} verifies to exist is removed
before the wrapper has a chance to run, the wrapper will throw an error.

This is currently problematic in local running, as the error will be
printed on top of the status and the application will exit.
This is due to how our async scheduling works, we make minimal use
of async functions and only those that have to be async are marked as such.

To fix this error printing there is a need for synchronization between
the status update function and the local runner, but this requires migrating
the status function to tokio runtime async functions.

Moreover this can also present itself to be a problem when running on Slurm.
The error will then be printed in the \texttt{slurm-[jobid].out} which is
very not user friendly.
